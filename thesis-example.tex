\documentclass[PhD]{PHlab-thesis}

\addbibresource{thesis-example.bib}

\newcommand*\Department中文{資訊工程學研究所}
\newcommand*\Department英文{Institute of Computer Science and Information Engineering}

\newcommand*\ThesisTitle中文{視覺化轉錄因子結合位點DNA\退{1}甲基化的新方法:MethylSeqLogo}
\newcommand*\ThesisTitle英文{MethylSeqLogo: A Novel Method for the Vizualization of DNA Methylation at Transcription Factor Binding Sites}
\newcommand*\ThesisNote中文{示例:其實徐翡曼是東京大學畢業的博士}% For real thesis omit, or use {初稿} etc.
\newcommand*\ThesisNote英文{Just an example.  Fei-Man actually graduated from Tokyo Univ.}% For real thesis omit, or use {draft} etc.

\newcommand*\Student中文{徐翡曼}
\newcommand*\Student英文{Fei-man Hsu}

\newcommand*\Advisor中文{賀保羅}
\newcommand*\Advisor英文{Paul Horton}

%% 果有共同指導老師可以用:
%% \newcommand*\CoAdvisorA中文{}
%% \newcommand*\CoAdvisorA英文{}
%% \newcommand*\CoAdvisorB中文{}
%% \newcommand*\CoAdvisorB英文{}


\newcommand*\YearMonth英文{July, 2022}
\newcommand*\YearMonth中文{111年7月}

\pagestyle{fancy}
\begin{document}


\newcommand*\Keywords英文{bioinformatics, genomics, string algorithms}
\newcommand*\Abstract英文{%
We introduce MethylSeqLogo, an extension of sequence logos to vizualize DNA methylation.
}


\newcommand*\Keywords中文{生命科學、基因組、字串演算法}
\newcommand*\Abstract中文{%
MethylSeqLogo...衍伸sequence logo的視覺化方法改善包括DNA甲基化的資訊。
}

\newcommand*\Acknowledgements{%
感謝我...}



\input{frontmatter}% 封面頁, 口委中英文簽名單, 誌謝, 中英文摘要, 論文目錄, 圖表目錄


%────────────────────  List of Symbols  ────────────────────
\renewcommand\nomgroup[1]{%
  \item[\bfseries
  \ifstrequal{#1}{A}{General}{%
  \ifstrequal{#1}{Z}{Gene/Protein Names}%
  }]}

\nomenclature[A]{$\lg$}{Logarithm base 2}
\nomenclature[A]{KL\ Divergence}{Kullback-Liebler Divergence}
\nomenclature[Z]{Myc}{MYC proto-oncogene}
\nomenclature[Z]{USF-1}{Upstream stimulatory factor 1}

\printnomenclature[5cm]

\newpage
\setcounter{page}{1}
\pagenumbering{arabic}



\chapter{Introduction}
DNA methylation is a major component of the epigenetic state of cells, as as such plays a critial role in the differentiation and maintenance of cell type.  Consequentially, aberrant DNA methylation plays a central role in cancer --- a manifestation of defective regulation of cell behavior.


\chapter{Related Works}
The characteristics of NuMT insertion sites have been analyzed in depth~\cite{TsujiJ_HortonP_NAR2012_Mammalian}.
Add your related works here.


\chapter{Method}
\section{Proposed Scheme}
Write a bunch of stuff here.

	
\chapter{Results}
Describe your results here.


\chapter{Discussion}
Discussion the significance or your results.


\section{Future Work}


\chapter{Conclusion}
Add your conclusions here.


\newpage
\AddToContents{References}
\printbibliography


\end{document}
